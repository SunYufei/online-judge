\subsection{基本数学运算}

\begin{table}[htbp]
	\centering
	\begin{tabular}{cc}
		\hline
		& C++ STL \\
		\hline
		求和 & accumulate(start, end, init) \\
		\hline
		\multirow{2}{*}{最大值} & max(x, y)  \\
		& max\_element(first, last[, cmp])  \\
		\hline
		\multirow{2}{*}{最小值} & min(x, y)  \\
		& min\_element(first, last[, cmp])  \\
		\hline
	\end{tabular}
\end{table}

\subsection{最大公约数与最小公倍数}

\paragraph{欧几里得算法(辗转相除法)}

求解最大公约数常用欧几里得算法,即若$a,b$均为整数$\gcd(a,b)=\gcd(b,a\%b)$。

\begin{lstlisting}[language=c++]
int gcd(int a, int b) {
	if (b == 0)
		return a;
	else
		return gcd(b, a % b);
}
\end{lstlisting}

最小公倍数$\textrm{lcm}(a,b)=\dfrac{ab}{\gcd(a,b)}$,在实际计算中$ab$有可能溢出,因此更恰当的写法是$\textrm{lcm}(a,b)=\dfrac{a}{\gcd(a,b)}*b}$。

\subsection{素数}

\subsubsection{素数的判断}

判断整数$n$是否为素数只需要判定$n$能否被$2,3,...,\lfloor\sqrt{n}\rfloor$中的一个整除即可,复杂度为$O(\sqrt{n})$。

\begin{lstlisting}[language=c++]
bool isPrime(int n) {
	if (n <= 1)
		return false;
	int sqr = (int)sqrt(n);
	for (int i = 2; i <= sqr; ++i)
		if (n % i == 0)
			return false;
	return true;
}
\end{lstlisting}

\subsubsection{素数表的获取}

使用Eratosthenes筛选法获取范围为$[1,n]$的素数表,首先需要枚举$[2,n]$的所有数,对每一个素数,筛去它的所有倍数,剩下的都是素数。

\begin{lstlisting}
vector<long> prime_array(long n) {
	int i, j;
	vector<bool> p(n + 1, false);
	vector<long> ret;
	for (i = 2; i <= n; ++i)
		if (p[i] == false) {
			ret.push_back(i);
			for (j = i + i; j < n; j += i)
				p[j] = true;
		}
	return ret;
}
\end{lstlisting}

\subsection{质因子分解}

\subsubsection{质因子分解}

质因子分解是指将正整数$n$写成一个或多个质数的乘积的形式,如$180=2^2\times3^2\times5$。显然,质因子分解最后归结到若干不同质数的乘积,可以先把素数表打印出来。

\textbf{注意:如果题目要求对1进行处理,视题目条件进行判定处理。}

\begin{lstlisting}
map<int, int> ans;	// 质因子映射
vector<long> prime = prime_array(n);	// 素数表
if (n == 1)	do_something();	// n = 1 情况
else {
	int sqr = (int)sqrt(n), i;
	for (i = 0; i < n && prime[i] <= sqr; ++i) {
		// 若 prime[i] 是 n 的因子
		if (n % prime[i] == 0) {
			ans[prime[i]] = 0;	// 计算质因子个数
			while (n % prime[i] == 0) {
				ans[prime[i]]++;
				n /= prime[i];
			}
		}
		if (n == 1)	// 及时退出
			break;
	}
	if (n != 1) // 若无法被根号 n 以内的质因子除尽
		ans[n] = 1;
}
\end{lstlisting}

\subsubsection{$n!$中质因子$p$个数}

例如:$6!=1\times2\times3\times4\times5\times6$,于是6!中有4个质因子2,2个质因子3。$n!$中有$\displaystyle{\left( \left\lfloor \frac{n}{p} \right\rfloor + \left\lfloor \frac{n}{p^2} \right\rfloor  + ...\right)}$个质因子$p$,时间复杂度为$O(\log n)$。

\begin{lstlisting}
int calc(int n, int p) {
	int ans = 0;
	while(n) {
		ans += n / p;
		n /= p;
	}
	return ans;
}
\end{lstlisting}

利用这个算法可以计算出\textbf{$n!$的末尾有多少个0}。由于末尾0的个数等于$n!$中因子10的个数,也等于$n!$中质因子5的个数,因此只需计算calc(n, 5)就可得到结果。

将此例推广到一般情况可得,$n!$中质因子$p$的个数等于$[1,n]$中$p$的倍数的个数$\displaystyle{\frac{n}{p}}$加上$\displaystyle{\frac{n}{p}!}$中质因子$p$的个数,因此可到递归版本的calc函数。

\begin{lstlisting}
int calc(int n, int p) {
	if (n < p)
		return 0;
	return n / p + calc(n / p, p);
}
\end{lstlisting}

\subsection{大整数运算}

\subsubsection{高精度计算}

浮点数高精度计算时使用double类型存储待计算数据进行计算,整数运算需要使用大整数。

\subsubsection{大整数的存储、比较与输出}

使用整数数组int d[1000]存储,\textbf{整数的高位存储在数组的高位,低位存储在数组的低位}。

\begin{lstlisting}
struct bign {
	int d[1000];
	int len;
	bign() {
		memset(d, 0, sizeof(d));
		len = 0;
	}
};
\end{lstlisting}

把整数按字符串读入时,需要在读入之后存储到d[]时反转一下。

\begin{lstlisting}
bign change(char str[]) {
	bign a;
	a.len = strlen(str);
	for (int i = 0; i < a.len; ++i)
		a.d[i] = str[a.len - i - 1] - '0';
	return a;
}
\end{lstlisting}

比较两个大整数大小的规则简单:先判断两者len的大小,若不相等,则以长的为大;如果相等,则从高位到低位进行比较,直到出现某一位不等,就可以判断两个数的大小。

\begin{lstlisting}
// a = b,0;a > b,1;a < b,-1
int compare(bign a, bign b) {
	if (a.len > b.len)		return 1;
	else if (a.len < b.len)		return -1;
	else {
		for (int i = a.len - 1; i >= 0; --i) {
			if (a.d[i] > b.d[i])	return 1;
			else if (a.d[i] < b.d[i]) return -1;
		}
		return 0;
	}
}
\end{lstlisting}

大整数输出时从最高位到最低位依次输出即可。

\begin{lstlisting}
void print(bign a) {
	for (int i = len - 1; i >= 0; --i)
		printf("%d", a.d[i]);
}
\end{lstlisting}

\subsubsection{大整数的四则运算}
\label{bignumber}

\paragraph{高精度加法}

对于其中的某一位,将该位上的两个数字和进位相加,得到的结果取个位数作为该位结果,取十位数作为新的进位。

\begin{lstlisting}
bign add(bign a, bign b) {
	bign c;
	int carry = 0, i, t;	// carry 为进位
	// 以较长的为边界
	for (i = 0; i < a.len || i < b.len; ++i) {
		t = a.d[i] + b.d[i] + carry;
		c.d[c.len++] = t % 10;	// 该位结果
		carry = t / 10;	// 新的进位
	}
	if (carry != 0)	// 最后进位不为 0
		c.d[c.len++] = carry;
	return c;
}
\end{lstlisting}

\paragraph{高精度减法}

对于其中的某一位,比较被减位和减位,如果不够减,向高位借1,如果够减,直接减。注意减法后\textbf{高位多余的0要忽视},但\textbf{保证结果至少一位数}。

\begin{lstlisting}
bign sub(bign a, bign b) {
	bign c; int i;
	// 以较长的为边界
	for (i = 0; i < a.len || i < b.len; ++i) {
		if (a.d[i] < b.d[i]) {	// 如果不够减
			a.d[i + 1]--;
			a.d[i] += 10;
		}
		c.d[c.len++] = a.d[i] - b.d[i];
	}
	// 去除高位的 0,同时至少保留一位最低位
	while (c.len-1 >= 1 && c.d[c.len - 1] == 0)
		--c.len;
	return c;
}
\end{lstlisting}

\paragraph{高精度与低精度乘法}

低精度指的是基本数据类型,如int等。对于某一位来说,取bign的某一位与int型整体相乘,再与进位相加。

\begin{lstlisting}
bign multiply(bign a, int b) {
	bign c;
	int carry = 0, i, t;	// carry 进位
	for (i = 0; i < a.len; ++i) {
		t = a.d[i] * b + carry;
		c.d[c.len++] = t % 10;
		carry = t / 10;
	}
	// 乘法的进位可能不止一位
	while (carry != 0) {
		c.d[c.len++] = carry % 10;
		carry /= 10;
	}
	return c;
}
\end{lstlisting}

\paragraph{高精度与低精度除法}

除法运算与小学所学相同。对于某一步操作:上一步的余数乘以10加上该步的位,得到该步临时的被除数,将其与除数比较。若不够除,则该位的商为0;若够除,则商为对应的商,余数为对应的余数。最后一步要减去高位多余的0,并保证结果至少一位。

\begin{lstlisting}
bign divide(bign a, int b, int &r) {
	bign c;
	// 被除数的每一位和商的每一位对应,先令长度相等
	c.len = a.len;
	for (int i = a.len - 1; i >= 0; --i) {
		// 和上一位遗留的余数组合
		r = r * 10 + a.d[i];	
		if (r < b)	// 不够除,该位为 0
			c.d[i] = 0;
		else {	// 够除
			c.d[i] = r / b;
			r = r % b;
		}
	}
	// 去除高位的 0,同时至少保留一位最低位
	while (c.len-1 >= 1 && c.d[c.len - 1] == 0)
		--c.len;
	return c;
}
\end{lstlisting}

\subsection{组合数计算}

组合数计算主要讨论$\textrm{C}_n^m$和$\textrm{C}_n^m\%p$两个问题。

对于$\textrm{C}_n^m$,可以使用定义式$\displaystyle {\textrm{C}_n^m=\frac{n!}{m!(n-m)!}}$直接计算,在$n=21$时超出long long范围;也可使用递推公式$\textrm{C}_n^m=\textrm{C}_{n-1}^{m}+\textrm{C}_{n-1}^{m-1},\textrm{C}_n^0=\textrm{C}_n^n=1$编写递归程序,注意重复计算问题,在$n=67,m=33$时溢出;推荐使用定义式变形$\displaystyle{\textrm{C}_n^m=\prod_{i=1}^{m}\frac{n-m+i}{i}}$来计算,复杂度为$O(m)$,在$n=62,m=31$时溢出:

\begin{lstlisting}
long long C(int n, int m) {
	long long ans = 1;
	for (int i = 1; i <= m; ++i)
		// 注意一定要先乘后除
		ans = ans * (n - m + i) / i;
	return ans;
}
\end{lstlisting}

一般来说,常见的情况是求$\textrm{C}_n^m\%p$,有四种方法,有各自的数据适合范围,一般第一种方法已经满足需求。

\textbf{方法一},$m\leq n\leq 1000$,$p$无限制,求解方法是求$\textrm{C}_n^m$的递推式变形

\begin{lstlisting}
int res[1010][1010] = {0};
int C(int n, int m, int p) {
	if (m == 0 || m == n)
		return 1;
	if (res[n][m] != 0)
		return res[n][m];
	res[n][m] = (C(n-1, m) + C(n-1, m-1)) % p;
	return res[n][m];
}
\end{lstlisting}

\textbf{方法二},Lucas定理,$m\leq n\leq 10^{18}$,$p$是素数

\begin{lstlisting}
int p;	// 全局 p 
int Lucas(int n, int m) {
	if (m == 0)
		return 1;
	return C(n % p, m % p)*Lucas(n/p, m/p) % p;
}
\end{lstlisting}
